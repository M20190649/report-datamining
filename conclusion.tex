To conclude we can say that we positioned the majority of customers using the localized CDR events generated by them with more than 1500 meters error.
The positioning of a customer using the CDR events generated by them, heavily depends on the cell map that is used for obtaining the location of the cell tower assigned to the CDR event. 
We found that OpenCelliD, the open-source cell map data has poor data quality for the Berlin region and is not recommended for accurate event positioning. 
However using a more accurate cell map, such as an internal data source from a telephone provider would improve the localization.
Other relevant factors that determine the accuracy of positioning are mobile usage habits, cellular network characteristics and the customer's geographical position.
To improve the positioning the shape and the characteristics of the customer's path can be considered to filter out points that are impossible with walking or other means of regular transportation. Taking the road network and traffic rules into account could also help positioning the customer. Also there could be more sophisticated distance and error measures defined for better results.
Examining and researching the characteristics of the cellular networks, such as direction of the antennas on the cell towers and the signal strength of the transmitting cell phone for each CDR event can also enable better positioning via cell tower triangulation method. 
