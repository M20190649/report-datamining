\documentclass{sig-alternate-05-2015}
%Hide ISBN from bottom
\makeatletter
\def\@copyrightspace{\relax}
\makeatother

\usepackage{subfiles}
\usepackage{nomencl}
\makenomenclature
\usepackage{hyperref}
\usepackage{footnote}
\usepackage{braket}
\usepackage{algorithm}
\usepackage[noend]{algpseudocode}
\makesavenoteenv{tabular}

\begin{document}
%Title
\title{Evaluation of mobile network events based geo-positioning}
\subtitle{[Extended Abstract]}

%Author(s)
\numberofauthors{1}
\author{
\alignauthor
Julia Hermann\\
       \affaddr{University of Trento}\\
       \email{julia.hermann@studenti.unitn.it}
}

%Abstract
\maketitle
%Date
\date{6 June 2018}
\begin{abstract}
In this paper we validate the usage of localized Call Detail Records data for customer positioning by comparing them to ground-truth data. The analysis shows that the majority of customers cannot be positioned given their CDR data with less than 1500 meters accuracy. This result depends on individual mobile usage habits, network characteristics and geographical position.

\end{abstract}
\keywords{big data, Spark, geo-positioning, GPS, CDR}
%Main parts
\section{Introduction}\subfile{intro}
\section{Problem Statement}\subfile{problem_statement.tex}
\section{Literature Review}\subfile{related_work.tex}
\section{Solution}\subfile{solution.tex}
\section{Proposed Approach}\subfile{approach.tex}
\section{Experimental Evaluation}\subfile{experiments.tex}
\section{Conclusion}\subfile{conclusion.tex}

\bibliographystyle{abbrv}
\bibliography{refs}
\end{document}
