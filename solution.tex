As the users tend to move while using their phones, the generated CDR observations have semantics of trajectories. CDR data points can be grouped as trajectories so that the characteristics of trajectories and the result of previous research already conducted in this field can be utilized for achieving more accurate positioning.

We compare the ground truth GPS trajectories to trajectories of CDR events for the same time window as shown in Figure \ref{fig:problem}. 
\begin{figure}[h]
    \centering
    \includegraphics[width=0.45\textwidth]{images/problem.png}
    \caption{Schematic illustration of the research problem}
    \label{fig:problem}
\end{figure}

To evaluate cellular network event based customer localization we calculate the Euclidean distance and the edit distance along with several statistics of the pairwise distances between ground truth GPS trajectories and mobile network event trajectories. The Haversine distance is used to define the spatial distance between points. In the mobile network event trajectory, the location information is obtained from cell map data set.


Distance measure on trajectories is defined similarly to \cite{encyclopedia, distance-def}:
\begin{definition}
Let $\mathcal{M}$ be a set of trajectories. A function $D :\mathcal{M} \times \mathcal{M} \rightarrow \mathcal{R}$ \nomenclature{$D$}{Distance measure defined on a set of trajectories.} is called a dissimilarity (distance) on $\mathcal{M}$ if for all $T_{1}, T_{2} \in \mathcal{M}$: 
\begin{itemize}
    \item $D(T_{1},T_{2}) \geqslant 0$
    \item $D(T_{1},T_{2}) = d(T_{2},T_{2})$
    \item $D(T_{1},T_{1}) = 0$
\end{itemize}
If all of these conditions are satisfied and $D(T_{1}, T_{2}) = 0 \Rightarrow  T_{1} = T_{2} $ is considered to be a symmetric. If
the triangle inequality is also satisfied, $D$ is a metric.
\end{definition}

The \textit{Euclidean distance} (or inverse similarity) measure between two trajectories is used in this analysis:
\[D(T,\Tilde{T}) = \sqrt{\sum_{i=1}^{n} (d(P_{i},\Tilde{P_{i}}))^{2}}\]

where $d((P_{i},\Tilde{P_{i}})$ refers to the spatial distance between two points given with the Haversine formula.

The Euclidean distance formula is implemented as presented in Algorithm \ref{algo:euclidean}.

\begin{algorithm}
\begin{algorithmic}
\caption{Euclidean distance function on trajectories} \label{algo:euclidean}
\Function{euclideanDist}{Traj tReal, Traj tApprox}
\State dEuclidean = 0
\State nPoints = len(tReal) 
\For{i \textbf{in} 0 to nPoints}
    \State dEuclidean = dEuclidean + power(haversine(     
    \State tReal[i], tApprox[i]) , 2)
\EndFor
\Return sqrt(dEuclidean)
\EndFunction
\end{algorithmic}
\end{algorithm}

We define the edit distance $D_{m,n}$ for two sequences $a = a_1\ldots a_n$ and $b = b_1\ldots b_m$ similarly to Levenshtein's formula \cite{edit_dist}:
\[
\begin{aligned}
\quad {\text{for}}\;1\leq i\leq m \\
D_{{i,0}}=\sum _{{k=1}}^{{i}}w_{{\mathrm  {del}}}(b_{{k}}) \\
\quad {\text{for}}\;1\leq j\leq n & \\
D_{{0,j}}=\sum _{{k=1}}^{{j}}w_{{\mathrm  {ins}}}(a_{{k}}) \\
\quad {\text{for}}\;1\leq i\leq m,1\leq j\leq n \\
D_{{i,j}}={\begin{cases}D_{{i-1,j-1}}&{\text{for}}\;a_{{j}}=b_{{i}}\\
\min {\begin{cases}D_{{i-1,j}}+w_{{\mathrm  {del}}}(b_{{i}})\\D_{{i,j-1}}+w_{{\mathrm  {ins}}}(a_{{j}})\\D_{{i-1,j-1}}+w_{{\mathrm  {sub}}}(a_{{j}},b_{{i}})\end{cases}}&{\text{for}}\;a_{{j}}\neq b_{{i}}\end{cases}} 
\end{aligned}
\]

In this research we want to determine distance of spatial trajectories therefore the cost function of the edit distance algorithm needs to be modified to incorporate the spatial distribution of trajectories. As an extension to traditional edit distance measure, we apply the cost functions from \cite{spatial_edit} as follows:
\begin{multline*}
w_{{\mathrm  {del}}} (b_{{k}}) = \biggl(\biggl[\frac{\sum \limits_{i=1}^n x_{i}^{b}}{n}-\frac{\sum \limits_{i=1, i \neq k}^n x_{i}^{b}}{n-1}\biggr]^2 \\+ \biggl[ \frac{\sum \limits_{i=1}^n y_{i}^{b}}{n}-\frac{\sum \limits_{i=1, i \neq k}^n y_{i}^{b}}{n-1}\biggr]^2\biggr)^{1/2}
\end{multline*}
\begin{multline*}
w_{{\mathrm  {ins}}} (a_{{j}}) = \biggl(\biggl[\frac{\sum \limits_{i=1}^n x_{i}^{b}}{n}-\frac{\sum \limits_{i=1}^n x_{i}^{b} + x_{j}^{a}}{n+1}\biggr]^2 \\ + \biggl[ \frac{\sum \limits_{i=1}^n y_{i}^{b}}{n}-\frac{\sum \limits_{i=1}^n y_{i}^{b} + y_{j}^{a}}{n+1}\biggr]^2\biggr)^{1/2}
\end{multline*}
\begin{multline*}
w_{{\mathrm  {sub}}} (a_{{j}}, b_{{k}}) = \biggl( \biggl[\frac{\sum \limits_{i=1}^n x_{i}^{b}}{n}-\frac{\sum \limits_{i=1, i \neq k}^n x_{i}^{b} + x_{j}^{a}}{n}\biggr]^2\\ + \biggl[\frac{\sum \limits_{i=1}^n y_{i}^{b}}{n}-\frac{\sum \limits_{i=1, i \neq k}^n y_{i}^{b} + y_{j}^{a}}{n}\biggr]^2 \biggr)^{1/2}
\end{multline*}
where $b_{{k}}$ and $a_{{j}}$ are points given as a pair of coordinates $(x_{{k}}^{b}, y_{{k}}^{b})$ and $(x_{{j}}^{a}, y_{{j}}^{a})$, respectively.

In CDR data each cell tower is referenced by its geographic coordinates, most commonly in the form of latitude and longitude. Therefore, traditional similarity measures in Euclidean space are not sufficient. The latitude, longitude values are need to be converted to Cartesian coordinates. In this research the EPSG:31468\footnote{For more details visit \url{https://epsg.io/31468}.} coordinate system is used, which can be used in some states of Germany, namely Bayern, Berlin, Niedersachsen and Schleswig-Holstein.

The edit distance formula is implemented as presented in Algorithm \ref{algo:edit}.

\begin{algorithm}
\begin{algorithmic}
\caption{Edit distance function on trajectories} \label{algo:edit}
\Function{euditDist}{Traj tReal, Traj tApprox}
\State n = len(tReal)
\State m = len(tApprox) 
\State dist = matrix(m, n)
\If {n == 0}
    \State sum = 0
    \For{i \textbf{in} 0 to m}
        \State sum = sum + $w_{del}$(b, i)
    \EndFor
    \Return sum
\EndIf

\If {m == 0}
    \State sum = 0
    \For{j \textbf{in} 0 to n}
        \State sum = sum + $w_{ins}$(a, j)
    \EndFor
    \Return sum
\EndIf

\If {n > 0 and m > 0}
   \For{i \textbf{in} 0 to m}
        \For{j \textbf{in} 0 to n}
            \If {b[i] == a[j]}
                \State dist[i, j] = dist[i - 1, j - 1]
            \Else
                \State dist[i, j] = \textbf{min}(dist[i - 1, j] + $w_{del}$(b, i), dist[i, j - 1] + $w_{ins}$(b, a, j), dist[i - 1, j - 1] + $w_{sub}$(b, a, i, j))
            \EndIf
        \EndFor
    \EndFor
    \Return dist[m - 1, n - 1]
\EndIf
\EndFunction
\end{algorithmic}
\end{algorithm}

It is important to note that the edit distance is a dynamic programming algorithm and is relatively time consuming with a time complexity of $\mathcal{O}(mn)$ where $m$ and $n$ are the lengths of the respective trajectories.

Given a set of distance observations with size $n$, we consider $\{d_{1}, d_{2}, ..., d_{n}\}$ as $n$ independent identically distributed (i.i.d) random variables, each corresponding to one randomly selected observation and having the distribution of the population. Then we can calculate the following descriptive statistics for the sample:
\begin{itemize}
    \item mean:
     \[ \overline {d} = \frac{\sum_{i=1}^{n} d_{i}}{n}\]
     \item standard deviation:
    \[ \sigma={\sqrt {\frac {\sum _{i=1}^{n}(d_{i}-{\overline {d}})^{2}}{n-1}}}\]
    \item sum:        
    \[s = \sum_{i=1}^{n} d_{i}\]
\end{itemize}

Note that $\{d_{1}, d_{2}, ..., d_{n}\}$ are the spatial distances between pairs of points from the trajectories. For each pair of points, $d_{i} = d(P_{i},\Tilde{P_{i}})$ and $n$ is the number of points in the trajectories.

For calculation of the descriptive statistics and the Euclidean distance the number of CDR and GPS observations need to match, which by the design of the data collection is guaranteed.

In this paper we do not attempt to reconstruct trajectories from CDR data considering the road network of Berlin. Instead we consider the sequence of positioned mobile network events together with the GPS observation and we determine the distance of the two trajectories based on the distances of individual pairs of points.



